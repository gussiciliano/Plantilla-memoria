\chapter{Diseño e implementación} % Main chapter title
Como se mencionó en la primera sección del documento, para este desarrollo se toman como input las pruebas de concepto y proyectos base hechos en el laborotario de software de la UNLa. A continuación se explica el foco de cada investigación por parte del equipo.

\section{Arquitectura del sistema}

%----------------------------------------------------------------------------------------

\section{Modelo de datos}

%----------------------------------------------------------------------------------------

\section{Modelado y confección del dispositivo }

%----------------------------------------------------------------------------------------

\subsection{Desarrollo base del backend y frontend}
Por parte de la capa lógica, Luciano Otegui y Guido Contento hicieron foco en generar la base del backend y del frontend para el trabajo. Se realizó el armado de casos de uso y diagrama de clases. Por otra parte, crearon la estructura mínima del sistema para una correcta interacción de usuarios finales. Finalmente, se definieron las vistas y estilos generales del sitio. Al igual que el caso anterior, el código base del backend [26] y del frontend [27] se alojaron en repositorios del dominio de la universidad.

\section{Desarrollo del backend}

%----------------------------------------------------------------------------------------

\section{Desarrollo del frontend}

%----------------------------------------------------------------------------------------

\section{Despliegue del sistema}

%----------------------------------------------------------------------------------------

\section{Integración del sistema completo}

%----------------------------------------------------------------------------------------
